% -*- mode: LaTeX; eval: (TeX-PDF-mode); TeX-engine: xetex; coding: utf-8-unix; TeX-master: t; ispell-local-dictionary: "brasileiro"; -*-

\documentclass[10pt,oneside]{article}
\usepackage{geometry}
\usepackage{fontspec}
\usepackage[T1]{fontenc}

\pagestyle{empty}
\geometry{letterpaper,tmargin=1in,bmargin=1in,lmargin=1in,rmargin=1in,headheight=0in,headsep=0in,footskip=.3in}

\setlength{\parindent}{0in}
\setlength{\parskip}{0in}
\setlength{\itemsep}{0in}
\setlength{\topsep}{0in}
\setlength{\tabcolsep}{0in}

% Name and contact information
\newcommand{\name}{Felipe Magno de Almeida}
\newcommand{\addr}{Rua Coronel Paulo Malta Rezende, 35/1708 - Barra da Tijuca - Rio de Janeiro-RJ}
\newcommand{\phone}{(21) 8242-8479}
\newcommand{\email}{eu@felipemagno.com.br}

%%%%%%%%%%%%%%%%%%%%%%%%%%%%%%%%%%%%%%%%%%%%%%%%%%%%%%%%%
% New commands and environments

% This defines how the name looks
\newcommand{\bigname}[1]{
	\begin{center}\fontfamily{phv}\selectfont\Huge\scshape#1\end{center}
}

% A ressection is a main section (<H1>Section</H1>)
\newenvironment{ressection}[1]{
	\vspace{4pt}
	{\fontfamily{phv}\selectfont\Large#1}
	\begin{itemize}
	\vspace{3pt}
}{
	\end{itemize}
}

% A resitem is a simple list element in a ressection (first level)
\newcommand{\resitem}[1]{
	\vspace{-4pt}
	\item \begin{flushleft} #1 \end{flushleft}
}

% A ressubitem is a simple list element in anything but a ressection (second level)
\newcommand{\ressubitem}[1]{
	\vspace{-1pt}
	\item \begin{flushleft} #1 \end{flushleft}
}

% A resbigitem is a complex list element for stuff like jobs and education:
%  Arg 1: Name of company or university
%  Arg 2: Location
%  Arg 3: Title and/or date range
\newcommand{\resbigitem}[3]{
	\vspace{-5pt}
	\item
	\textbf{#1}---#2 \\
	\textit{#3}
}

% A resbigitem_1 is a complex list element for stuff like jobs and education:
%  Arg 1: Name of company or university
%\newcommand{\resbigitem_1}[1]{
%	\vspace{-5pt}
%	\item
%	\textbf{#1}
%}

% This is a list that comes with a resbigitem
\newenvironment{ressubsec}[3]{
	\resbigitem{#1}{#2}{#3}
	\vspace{-2pt}
	\begin{itemize}
}{
	\end{itemize}
}

% This is a list that comes with a resbigitem
\newenvironment{ressubsec_1}[1]{
	\vspace{-5pt}
	\item
	\textbf{#1}
	\vspace{-2pt}
	\begin{itemize}
}{
	\end{itemize}
}

\newenvironment{ressubsecvalue}[4]{
	\resbigitem{#1}{#2}{#3} \\
	{\footnotesize #4}
	\vspace{-2pt}
	\begin{itemize}
}{
	\end{itemize}
}

% This is a simple sublist
\newenvironment{reslist}[1]{
	\resitem{\textbf{#1}}
	\vspace{-5pt}
	\begin{itemize}
}{
	\end{itemize}
}



%%%%%%%%%%%%%%%%%%%%%%%%%%%%%%%%%%%%%%%%%%%%%%%%%%%%%%%%%
% Now for the actual document:

\begin{document}

\fontfamily{ppl} \selectfont

% Name with horizontal rule
\bigname{\name}

\vspace{-8pt} \rule{\textwidth}{1pt}

\vspace{-1pt} {\small\itshape \addr \hfill \phone; \email}

\vspace{8 pt}


%%%%%%%%%%%%%%%%%%%%%%%%
\begin{ressection}{Objetivo}

	\resitem{Professor de C \& C++}

\end{ressection}


%%%%%%%%%%%%%%%%%%%%%%%%
\begin{ressection}{Educa\c{c}\~ao}

	\resitem{Universidade Estadual de Campinas (Unicamp) - Campinas, SP - Ci\^encia da Computa\c{c}\~ao.}
		%\ressubitem{Ingresso: 2003/Incompleto}
	%\end{ressubsec}

\end{ressection}


%%%%%%%%%%%%%%%%%%%%%%%%
\begin{ressection}{Experi\^encia}

  \begin{ressubsec}{TECGRAF}{Rio de Janeiro, RJ}{Analista de Sistemas: Junho de 2010--Atual}
    \ressubitem{Trabalho de correção e design de projetos baseados em componentes em sistemas distribuídos}
    \ressubitem{Uso intensivo de tecnologia CORBA}
    \ressubitem{Contato com equipe de desenvolvimento Petrobrás para suporte das bibliotecas desenvolvidas}
  \end{ressubsec}

	\begin{ressubsec}{TQTVD}{Rio de Janeiro, RJ}{Analista de Sistemas: Janeiro de 2009--Junho de 2010}
      \ressubitem{Correções de concorrência, memory leaks e type subverting em código de desenvolvimento
        terceirizado, adaptado para uso na empresa.}
      \ressubitem{Trabalho em grupo com correções em middleware e sistema multiplataforma.}
      \ressubitem{Desenvolvimento de aplicações embarcadas e utilização de plataformas de memória restrita.
        Criação de middleware para execução de código alto-nível em plataformas embarcadas}
      \ressubitem{Reescrita de interpretador NCL, linguagem utilizada em TV digital, multi-plataforma,
        unit-tests e genérica.}
      \ressubitem{Integração de linguagem Python com middleware de TV digital.}
	\end{ressubsec}

	\begin{ressubsec}{Synergy Tecnologia}{S\~ao Paulo, SP}{Analista de Sistemas: Janeiro de 2004--2009}
		\ressubitem{Mantido, redesenhado e programado o projeto MailIntercept (http://www.mintercept.com). Tendo como clientes Missile Defense Agency e Raytheon.}
		\ressubitem{Profundo conhecimento dos padr\~oes de MIME, SMTP, RFC[2]822. Como tamb\'em do formato TNEF de e-mail do Exchange Server.
		  Tendo escrito v\'arios parsers para esses padr\~oes, bem como clientes que utilizam asynchronous I/O de SMTP e HTTP.}
	\end{ressubsec}

	\begin{ressubsec}{Expertise Solutions Consultoria em Inform\'atica}{Campinas, SP}{Consultor e Sócio: Maio de 2006--at\'e o momento}
		\ressubitem{Professor de uma turma sobre C++ para iniciantes dada no Laborat\'orio do Instituto de
		Computa\c{c}\~ao da Unicamp.}
		\ressubitem{Professor de duas turmas no Laboratório do Instituto de Computa\c{c}\~ao de curso intermediário de C++, com total de 137 alunos.}
	\end{ressubsec}

	\begin{ressubsec}{Z80}{Campinas, SP}{Programador C para BREW: Outubro de 2003--2004}
		\ressubitem{Programação de jogos para celular.}
	\end{ressubsec}

%%%%%%%%%%%%%%%%%%%%%%%%
\newpage
%%%%%%%%%%%%%%%%%%%%%%%%

	\begin{ressubsec_1}{Boost Libraries}
		\ressubitem{Constante nos agradecimentos de tr\^es bibliotecas: wave, pointer container e multi\_index}
		\ressubitem{Constante nos agradecimentos da proposta de biblioteca de rede ao comite de padroniza\c{c}\~ao do C++
		  (http://www.open-std.org/jtc1/sc22/wg21/docs/papers/2006/n2054.pdf), da biblioteca recentemente adicionada asio.}
		\ressubitem{Trabalhando com peer review de código da cppgui e engine de layout baseado na adam \& eve da Adobe
		  junto a Boost Libraries, para desenvolvimento de uma biblioteca com maior abrangência e qualidade possíveis.}
		\ressubitem{Ativo na mailing list de desenvolvimento, dando opiniões, enviando patches}
	\end{ressubsec_1}

	\begin{ressubsec_1}{Javabind}
	  \ressubitem{Autor de biblioteca para integração de código Java e C++}
          \ressubitem{Uso de preprocessor e template metaprogramming}
          \ressubitem{Criação (por preprocessor metaprogramming) de classes estáticas que representam
            as mesmas em Java, para checagem estática e performance}
          \ressubitem{Geração de Class File e bytecode}
	  \ressubitem{Código em git disponível em https://github.com/felipealmeida/javabind}
	\end{ressubsec_1}
	
	\begin{ressubsec_1}{mORBid}
	  \ressubitem{Autor de biblioteca de protocolo CORBA}
          \ressubitem{Uso de metaprogramação para geração de parsers e geradores a partir de gramática GIOP}
	  \ressubitem{Código em git disponível em https://github.com/felipealmeida/mORBid}
	\end{ressubsec_1}

	\begin{ressubsec_1}{5$^{\underline{o}}$ Encontro Nacional de Programadores de C e C++}
	  \ressubitem{Palestrante sobre Linguagens Embutidas de Domínio Específico.}
	\end{ressubsec_1}

\end{ressection}

\begin{ressection}{Conhecimentos}

	\begin{reslist}{Linguagens de Programa\c{c}\~ao:}

		\ressubitem{Proficiente em C, C++, Assembly 80x86/386+ e MIPS, SQL, HTML, \LaTeX, UNIX Shells, TCP, sockets, XML, postgresql
          , OpenSSL, Boost, STL, Qt, Gtk, pthreads, Win32, Win64, CMake, Linux, Windows, threads, BD
          , Design Patterns, CMake, svn, subversion, cvs, Mercurial, regexp, BJam, Boost.Build, CORBA, CMake}
		\ressubitem{Familiar com Java e C\#.}
		\ressubitem{Bastante interessado no futuro e presente de C++.}
        \ressubitem{Autodidata - Com biblioteca contendo vários livros de computação, com mais de 12 livros só de C++. Entre outros de compiladores, influ\^encia da tipagem
		  no processo de design de software, provabilidade de corretude e etc.}
		\ressubitem{Este curriculum \'e feito em \LaTeX.}
		\ressubitem{Inglês fluente.}

	\end{reslist}

\end{ressection}


%%%%%%%%%%%%%%%%%%%%%%%%
\begin{ressection}{Outros}
	\resitem{Autor de diversos materiais sobre C++, de iniciantes
      at\'e para programadores avan\c{c}ados, focando em Software
      Design.}
    \resitem{LinkedIn profile: http://br.linkedin.com/in/felipealmeida}
\end{ressection}


\end{document}
