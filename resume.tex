% -*- mode: LaTeX; eval: (TeX-PDF-mode); TeX-engine: xetex; coding: utf-8-unix; TeX-master: t; ispell-local-dictionary: "brasileiro"; -*-

\documentclass[10pt,oneside]{article}
\usepackage{geometry}
\usepackage{fontspec}
\usepackage[T1]{fontenc}
\usepackage{wrapfig}
\usepackage{eso-pic}
\usepackage{calc}
\usepackage{hyperref}
\usepackage{graphicx}
\pagestyle{empty}
\geometry{letterpaper,tmargin=1cm,bmargin=1in,lmargin=1in,rmargin=1in,headheight=0in,headsep=0in,footskip=.3in}

\newcommand\AtPageUpperRight[1]{\AtPageUpperLeft{%
   \makebox[\paperwidth-0.5cm][r]{#1}}}

\AddToShipoutPicture{%
\AtPageUpperRight{%
\raisebox{-1\height-0.5cm}{\makebox[1\width][r]{
 \def\svgwidth{3cm}
 \input{logo.pdf_tex}
}}
}}


\setlength{\parindent}{0in}
\setlength{\parskip}{0in}
\setlength{\itemsep}{0in}
\setlength{\topsep}{0in}
\setlength{\tabcolsep}{0in}

% Name and contact information
\newcommand{\name}{Felipe Magno de Almeida\\Expertise Solutions}
\newcommand{\addr}{}
\newcommand{\phone}{\url{https://expertise.dev}}
\newcommand{\email}{\href{mailto:felipe@expertise.dev}{\nolinkurl{felipe@expertise.dev}}}

%%%%%%%%%%%%%%%%%%%%%%%%%%%%%%%%%%%%%%%%%%%%%%%%%%%%%%%%%
% New commands and environments

% This defines how the name looks
\newcommand{\bigname}[1]{
	\begin{center}\fontfamily{phv}\selectfont\Huge\scshape#1\end{center}
}

% A ressection is a main section (<H1>Section</H1>)
\newenvironment{ressection}[1]{
	\vspace{4pt}
	{\fontfamily{phv}\selectfont\Large#1}
	\begin{itemize}
	\vspace{3pt}
}{
	\end{itemize}
}

% A resitem is a simple list element in a ressection (first level)
\newcommand{\resitem}[1]{
	\vspace{-4pt}
	\item \begin{flushleft} #1 \end{flushleft}
}

% A ressubitem is a simple list element in anything but a ressection (second level)
\newcommand{\ressubitem}[1]{
	\vspace{-1pt}
	\item \begin{flushleft} #1 \end{flushleft}
}

% A resbigitem is a complex list element for stuff like jobs and education:
%  Arg 1: Name of company or university
%  Arg 2: Location
%  Arg 3: Title and/or date range
\newcommand{\resbigitem}[3]{
	\vspace{-5pt}
	\item
	\textbf{#1}---#2 \\
	\textit{#3}
}

% A resbigitem is a complex list element for stuff like jobs and education:
%  Arg 1: Name of company or university
%  Arg 2: Location
%  Arg 3: Title and/or date range
\newcommand{\resonelinebigitem}[2]{
	\vspace{-5pt}
	\item
	\textbf{#1}---#2
}

% A resbigitem_1 is a complex list element for stuff like jobs and education:
%  Arg 1: Name of company or university
%\newcommand{\resbigitem_1}[1]{
%	\vspace{-5pt}
%	\item
%	\textbf{#1}
%}

% This is a list that comes with a resbigitem
\newenvironment{ressubsec}[3]{
	\resbigitem{#1}{#2}{#3}
	\vspace{-2pt}
	\begin{itemize}
}{
	\end{itemize}
}
\newenvironment{resonelinesubsec}[2]{
	\resonelinebigitem{#1}{#2}
	\vspace{-2pt}
	\begin{itemize}
}{
	\end{itemize}
}

% This is a list that comes with a resbigitem
\newenvironment{ressubsec_1}[1]{
	\vspace{-5pt}
	\item
	\textbf{#1}
	\vspace{-2pt}
	\begin{itemize}
}{
	\end{itemize}
}

\newenvironment{ressubsecvalue}[4]{
	\resbigitem{#1}{#2}{#3} \\
	{\footnotesize #4}
	\vspace{-2pt}
	\begin{itemize}
}{
	\end{itemize}
}

% This is a simple sublist
\newenvironment{reslist}[1]{
	\resitem{\textbf{#1}}
	\vspace{-5pt}
	\begin{itemize}
}{
	\end{itemize}
}



%%%%%%%%%%%%%%%%%%%%%%%%%%%%%%%%%%%%%%%%%%%%%%%%%%%%%%%%%
% Now for the actual document:

\begin{document}

\fontfamily{ppl} \selectfont

% Name with horizontal rule
\bigname{\name}

\vspace{-8pt} \rule{\textwidth}{1pt}

\vspace{-1pt} {\small\itshape \addr \hfill \phone; \email}

\vspace{8 pt}

%%%%%%%%%%%%%%%%%%%%%%%%
\begin{ressection}{Social}
  \resitem{Email: \href{mailto:felipe@expertise.dev}{\nolinkurl{felipe@expertise.dev}}}
  \resitem{Linkedin: \url{https://br.linkedin.com/in/felipealmeida}}
  \resitem{Github: \url{https://github.com/felipealmeida}}
  \resitem{Github: \url{https://github.com/expertisesolutions}}
  \resitem{EFL Git: \url{https://git.enlightenment.org}}
\end{ressection}


%%%%%%%%%%%%%%%%%%%%%%%%
\begin{ressection}{Specialization}
        \resitem{Linux Embedded Systems development with C, C++,
          Assembly and other higher-level languages}
        \resitem{IoT in Linux, RTOS or bare metal with C, C++ and Assembly}
	\resitem{GUI development in C, C++, JavaScript, Python and C\#}
        \resitem{Contiki, FreeRTOS, Zephyr and Linux driver development}
        \resitem{Mobile appllication development}
        \resitem{IPTV and ISDB-T/DVB-S development for Set-Top-Boxes}
%        \resitem{Distributed systems with high-performance with CORBA
%          and DBus}
\end{ressection}


%%%%%%%%%%%%%%%%%%%%%%%%
\begin{ressection}{Presentations}
  \resbigitem{Native Floripa 2018}{Florianopolis -
    Brazil}{C++ value-based programming}
  \resbigitem{Native Floripa 2017}{Florianopolis -
    Brazil}{Integrating generators EDSL's for Spirit X3}
  \resbigitem{Enlightenment Developer Days 2016}{Paris -
    France}{Promises and EFL Data Model \\ EFL JavaScript \& C++ bindings}
  \resbigitem{CppCon 2015}{Bellevue - WA -  USA}{Integrating generators EDSL's for Spirit X3}
  \resbigitem{12$^{\underline{o}}$ National C++ Meeting 2015}{Rio de Janeiro -
    Brazil}{Developing Graphical Interfaces in C++ using EFL C++ bindings}
  \resbigitem{C++Now 2015}{Aspen - CO - USA}{Integrating generators EDSL's for Spirit X3}
  \resbigitem{Enlightenment Developer Day 2014}{Dusseldorf -Germany}{EFL bindings for C++
    and JavaScript}
  \resbigitem{10$^{\underline{o}}$ National C++ Meeting 2013}{Rio de Janeiro -
    Brazil}{Interoperating C++ and Java using C++ metaprogramming}
  \resbigitem{5$^{\underline{o}}$ National C++ Meeting 2008}{São Paulo -
    Brazil}{Creating Embedded Domain Specific Languages for Performance}
\end{ressection}

%%%%%%%%%%%%%%%%%%%%%%%%
\newpage
\begin{ressection}{Projects}

  \begin{resonelinesubsec}{Gstreamer}{}
    \ressubitem{C and C++ development of sources and demuxers for
      proprietary IP camera protocol}
    \ressubitem{Development of sinks for AWS S3 upload}
    \ressubitem{Integration of gstreamer pipelines with AWS
      Rekognition service}
    \ressubitem{Android applications with Gstreamer}
  \end{resonelinesubsec}

  \begin{resonelinesubsec}{Window Compositor and GUI library}{}
    \ressubitem{Modern C++ development with goal in performance and
      battery usage}
    \ressubitem{Compositor implements Wayland Server protocol and
      composites Wayland clients}
    \ressubitem{GUI library uses Vulkan as backend for lower-level
      graphics API with better performance optimization opportunities}
    \ressubitem{GUI library and compositor uses multithreading allowed
      by Vulkan API for lower latency and better usage of CPU resources}
  \end{resonelinesubsec}
  
  \begin{resonelinesubsec}{EFL GUI-library}{}
    \ressubitem{Development of a code generator that generates C++
      code that integrates the
      \href{https://developers.google.com/v8/}{V8} library with the
      EFL C API}
    \ressubitem{Development of a code generator that generates C++
      code that integrates modern C++11 with the
      EFL C API}
    \ressubitem{Development of a code generator that generates C\#
      code that integrates modern C++11 with the
      EFL C API}
  \end{resonelinesubsec}

  % \newpage
  % \begin{resonelinesubsec}{Clang pre-condition and post-condition
  %     syntax support}{}
  %   \ressubitem{Development, in progress, of Clang suspport for pre
  %     and post-conditions to Clang}
  %   \ressubitem{Generation of \href{https://coq.inria.fr/}{coq},
  %     static assistant proof, code so users can make correctness
  %     proofs with the use of a proof assistant with Hoare Logic, or
  %     equivalent, theorems}
  % \end{resonelinesubsec}

  \begin{resonelinesubsec}{Digital TV Interactivity Middleware}{}
    \ressubitem{Development of a C++ Middleware to run interactivity
      systems in Digital TV standards for embedded systems}
    \ressubitem{Development of high performance filters in C++ for
      Transport Streams, DSM-CC implementation and other Digital TV
      related technologies}
  \end{resonelinesubsec}

  \begin{resonelinesubsec}{Mail System}{}
    \ressubitem{Development of SMTP libraries for high-performance in
      C++ using Boost and Asynchronous I/O}
    \ressubitem{Development of parsers using EBNF format in Embedded
      Domain Specific Language in Spirit}
    \ressubitem{Actively participated in Boost.Spirit community for years}
  \end{resonelinesubsec}

  \begin{resonelinesubsec}{CORBA}{}
    \ressubitem{Development of experimental C++ ORB for high
      performance while leveraging on Template meta-programming}
    \ressubitem{Meta-programming was able to move information from runtime to compile-time, avoiding unnecessary
      indirections and memory reads for parsing}
    \ressubitem{Involvement in develeopment of Authentication and
      Security System for CORBA standard}
    \ressubitem{Protocol for authentication used in brazilian oil companies}
  \end{resonelinesubsec}

  
\end{ressection}

\end{document}
